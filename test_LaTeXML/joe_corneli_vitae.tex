\documentclass[12pt,article]{article}

\renewcommand{\refname}{}%
\usepackage{color}

\catcode`\_=11\relax
\newcommand\email[1]{\_email #1\q_nil}
\def\_email#1@#2\q_nil{
  \href{mailto:#1@#2}
    {{\emailfont\detokenize{#1}\emailampersat\detokenize{#2}}}
}
\newcommand\emailfont{\rmfamily}
\newcommand\emailampersat{{\color{black}\small@}}
\catcode`\_=8\relax  

\PassOptionsToPackage{hyphens}{url}
\usepackage[breaklinks,linkbordercolor=cyan, pdfborderstyle={/S/U/W .5}]{hyperref}

\makeatletter
\g@addto@macro{\UrlBreaks}{\UrlOrds}
\makeatother

\begin{document}

\thispagestyle{plain}

\begin{center}
{\Large \bfseries  Joseph Angus Corneli \\}
Department of Computing \\
Goldsmiths College \\
New Cross, London \\
SE14 6NW, UK \\

\email{j.corneli@gold.ac.uk}
\end{center}

\section*{Education}

\paragraph{The Open University}  \hspace{.1in} \\
Doctor of Philosophy (Computing), 2014 (expected; submitted July, 2013) 

\smallskip

Thesis: \emph{\href{http://metameso.org/~joe/thesis-outline.html}{Peer
    produced peer learning: A mathematics case study}} (Supervisors:
Prof. Peter Scott and
Dr. Alexander Mikroyannidis).

\paragraph{New College of Florida}  \hspace{.1in} \\
Bachelor of Arts (Mathematics), 2002
\smallskip

Thesis: \emph{Double bubbles in spaces of constant curvature}
(Advisor: Prof. Patrick T. McDonald).

\section*{Professional Experience}

\paragraph{Goldsmiths College, University of London} \hspace{.1in} \\
Researcher in Computational Creativity.  Working with
partners to develop novel concept invention techniques and
assessment methodologies.  Applying concept invention
techniques to tasks in mathematical creativity.  Writing
and presentation of research papers at conference and
journal level.  (2013-2016, current position)

\paragraph{Knowledge Media Institute at The Open University} \hspace{.1in} \\
Research Assistant in the Digital Environment for Cultural
Interfaces; Promoting Heritage, Education and Research (DECIPHER) EU
project.  Responsibilities included PHP (Drupal) and Lisp programming,
and contributions to Semantic Web integration. (2013)

\paragraph{PlanetMath.org, Ltd.} \hspace{.1in} \\
Member of the Board of Directors of the US-based nonprofit
PlanetMath.org, Ltd., which runs a popular community-produced
mathematics website serving learners at all levels.  Key
accomplishments include strategic planning, research, and technical
work.  (2008-ongoing)

\paragraph{Cycorp, Inc.}  \hspace{.1in} \\
Associate Member of the Technical Staff at a well-known AI firm.
Devised programmatic methods for cleaning a large artificial
intelligence knowledge base. (6/2003-8/2003)

\section*{Teaching Experience}

\paragraph{Peer-2-Peer University} \hspace{.1in} \\
Designed and facilitated ``DIY Math''; ``Mathematics for
Game Designers''; ``Open Governance and Learning'' (co-organized with
Marisa Ponti); and ``Shaping P2PU''.  (10/2010--5/2011)

\paragraph{University of Texas at Austin} \hspace{.1in} \\
Teaching Assistant for Calculus I and II; Facilitated
weekly problem sessions, graded papers, and held office
hours. (6/2002-8/2002, 2/2003-5/2003)

\section*{Contributions to Free/Open Source Software}

\paragraph{Planetary}  \hspace{.1in} \\
\url{https://github.com/KWARC/planetary} \smallskip

Planetary is the software system powering PlanetMath, based
on Drupal 7, \LaTeX ML, Virtuoso and other free/open components.  I am
one of the main developers. (2010-ongoing)

\paragraph{Arxana}  \hspace{.1in} \\
\url{http://repo.or.cz/w/arxana.git} \smallskip

Arxana is a hypertext system written in literate Emacs Lisp,
with Common Lisp extensions.  I am the main developer. (2005-ongoing)

\nocite{*}

\section*{Peer Reviewed Publications}
\begingroup
\bibliographystyle{unsrt}
\bibliography{corneli-citations}
\endgroup

% \section*{Books}
% \begingroup
% \renewcommand{\section}[2]{}%
% \bibliographystyle{unsrt}
% \bibliography{corneli-books}
% \endgroup

\end{document}


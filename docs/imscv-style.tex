% arara: pdflatex
% arara: biber
% arara: pdflatex
% arara: pdflatex
\documentclass[letterpaper]{article}
\usepackage{imscv}

\titlespacing*{\subsection}{0pt}{2.5ex plus 1ex minus .2ex}{2.5ex plus 1ex minus .2ex}

\usepackage{mdframed}
\usepackage{alltt}
\addbibresource{imsBooks.bib}
\addbibresource{\jobname.bib}
\usepackage{filecontents}
\begin{filecontents}{\jobname.bib}
%% BibTeX file - none needed here
\end{filecontents}

\begin{document}

\title{Discussion of the ``IMS CV style'' (imscv.sty)}
\author{Joe Corneli}

\maketitle

\section{Introduction}

The current documents describes biographic records written in the
``IMS CV style'' that is, documents designed to be compiled using
imscv.sty.  This style is designed to produce nice-looking PDFs, but
also to store fielded data that can be easily converted to other
formats (for example, to HTML for display on the web).  In addition to
simple ``biographical'' and ``membership'' information, these files
typically also include ``bibliographic'' information relevant to the
life of the person described.  This information is recorded in
Bib\TeX\ records that are included within the preamble of
\LaTeX\ file, via {\tt filecontents}.  In standard usage, the
``biographical'' information appears next, followed by an inclusion of
the bibliographic records into the text.


\section{Compiling}

The beginning of the preamble of documents written to be compiled with
the imscv.sty style file looks like this:

\begin{verbatim}
% arara: pdflatex
% arara: biber
% arara: pdflatex
% arara: pdflatex
\documentclass[letterpaper]{article}
\usepackage{imscv}
\addbibresource{imsBooks.bib}
\addbibresource{\jobname.bib}
\usepackage{filecontents}
[...]
\end{verbatim}

They can be compiled either with the complex command {\tt pdflatex
  \emph{filename} \&\& biber \emph{filename} \&\& pdflatex
  \emph{filename} \&\& pdflatex \emph{filename}} or with the single
command {\tt arara \emph{filename}} for users who have {\tt arara}
installed.  (For example, it is part of texlive.)  For uniformity,
filenames are written without suffix.

\subsection*{Example}

\begin{alltt}
{\tt pdflatex public__john_q \&\& biber public__john_q \&\& pdflatex\(\rightarrow\)
    public__john_q \&\& pdflatex public__john_q}
\end{alltt}

or, more simply:

\begin{alltt}
arara public__john_q
\end{alltt}

\section{Biblographic records}

%% A common pattern in the legacy sources for is to store a
%% ``bibliographic-style'' reference with the following form:

%% \begin{alltt}
%% :BioCat bibtype person__id:N
%%         Title goes here
%%         Author goes here
%%         Howpublished information goes here 
%%         Year goes here
%%         URL goes here, if any
%%         :ref <t>Title repeated here</t>. <j>Name of journal</j>,\(\rightarrow\)
%%             Vol. <vol>X</vol>, <n>Y</n> other information not\(\rightarrow\)
%%             fielded pp. <pp>P-Q</pp>
%% \end{alltt}

With the new implementation, bibliography-style data is turned into
standard Bib\TeX\ records, included via {\tt filecontents} and
referred to in the appropriate section of the member bio.
%  Thus, the data above would be transformed into the
%following record:\footnote{(In practice the {\tt :ref} data typically
%  repeats the {\tt Howpublished}, sometimes with slightly different
%  formatting or information.)}
This example entry should look familiar (other bibtypes are supported as well):

\begin{alltt}
@journal\{person__id:N,
year=\{Year goes here\},
author=\{Author goes here\},
title=\{Title goes here\},
journal=\{Name of journal\},
volume=\{X\},
number=\{Y\},
pages=\{P-Q\},
url=\{URL goes here, if any\}\}
\end{alltt}

This will be brought into the \LaTeX\ file in the appropiate
subsection, like this:

\begin{alltt}
\verb|\|subsection\verb|*{|\emph{BioCat Heading}\verb|}}|
\emph{\textbackslash}\emph{BioCatCite}\verb|{|person__id:N\verb|}|
\end{alltt}

where {\tt \textbackslash \emph{BioCatCite}} is one of several
synonyms that are introduced for the usual \verb|\cite| command.  This
helps support automated scanning of the files.  The {\tt \emph{BioCat
    Heading}} is not expected to be processed automatically, so it
mainly serves as a visual guide and can be written in the plural or
singular form.

\bigskip

\begin{mdframed}
\begin{addmargin}[0em]{6em}
\begin{itemize}
\item {\tt \emph{\textbackslash BioCatCite}} $\in$ 
\{
\verb|\Biography|,
\verb|\Symposium|,
\verb|\Archive|,
\verb|\Autobiography|,
\verb|\ArtExhibition|,
\verb|\CollectedWorks|,
\verb|\SelectedWorks|,
\verb|\DVD|,
\verb|\Endowment|,
\verb|\Festschrift|,
\verb|\Memoir|,
\verb|\OralHistory|,
\verb|\DeathNotice|,
\verb|\Obituary|,
\verb|\InMemoriam|\}
\item {\tt \emph{BioCat Heading}} $\in$ \{Biography, Biographies,
Symposium, Symposia,
Archive, Archives,
\ldots ,
Obituary, Obituaries,
In Memoriam\} 
\end{itemize}
\end{addmargin}
\end{mdframed}

\bigskip

\subsection*{Example}

The following Bib\TeX\ records would be stored in the file
\verb|public__john_q.tex| inside a \verb|filecontents| block
in the document's preamble --

\begin{verbatim}
\begin{filecontents}{\jobname.bib}
[... some lines omitted ...]
@article{public__john_q:13,
  title={{J}ohn {Q}. {P}ublic visits a {L}and called {U}qbar},
  author={Haslam, Silas},
  journal={Revista Canadiense de Estudios Hisp{\'a}nicos},
  pages={275--280},
  year={2001},
  publisher={JSTOR}
}

@misc{public__john_q:14,
year={2002},
author={Anonymous},
title={{U}ntitled},
}
\end{filecontents}
\end{verbatim}

\ldots and would be brought into the main \LaTeX\ portion of the text
later on, like this:

\begin{verbatim}
\subsection*{Obituary}
\Obituary{public__john_q:13}       

\subsection*{In Memoriam}
\InMemoriam{public__john_q:14}
\end{verbatim}

Note: as mentioned in the introduction, in practice, the other
elements of the record, described in the sections below, will appear
before the bibliographic records are incorporated.

%\newpage

\section{Basic membership and biographical data}

These commands record straightforward information about the membership
and biography.  These commands each take one argument, which should be
self-explanatory:

\begin{mdframed}
\verb|\Name{NAME}|\quad
\verb|\ID{ID}|\quad
\verb|\AltID{ALT-ID}|\quad
\verb|\DOB{DOB}|

\verb|\DOD{DOD}|\quad
\verb|\DisplayName{SHORT-NAME}|\quad
\verb|\AltName{OTHER-NAME}|

\verb|\BirthPlace{PLACE}|\quad \verb|\Deceased|
\end{mdframed}

As indicated, the command \verb|\Deceased| takes \emph{zero} arguments
and should be placed after \verb|\Name{...}|, only in the case in
which the record describes a member who is deceased.  Note: The format
of the argument passed to \verb|\Name{...}| should be
\texttt{\emph{family name}, \emph{given name(s)}}.

\subsection*{Example}

\begin{verbatim}
[... preamble, including bibliography data as above, omitted ...]
\begin{document}

\Name{Public, John Q.} \Deceased

\ID{public__john_q}
\AltID{public__john}

\DisplayName{Public, John}
\AltName{John Doe}

\DOB{1963-11-23}
\DOD{2001-09-10}
\BirthPlace{USA}
[...]
\end{verbatim}

\section{Links}


\begin{mdframed}
\texttt{\textbackslash Homepage\{URL\}} \quad
\texttt{\textbackslash Email\{EMAIL-ADDRESS\}} \quad
\texttt{\textbackslash Profile\{URL\}\{TEXT\}}
\end{mdframed}

Note that \verb|\Email{...}| is not something that every user wishes
to publish on the open web, so while this field is supported at the
user's option, it is not in any way required.  

\subsection*{Example}

\begin{verbatim}
[...]
\Homepage{http://www.jqpublicblog.com}
\Email{estate_of_john_public@gmail.com}

\subsection*{Profiles}
\Profile{http://en.wikipedia.org/wiki/John_Q._Public}{Wikipedia}
\Profile{http://en.wiktionary.org/wiki/John_Q._Public}{Wiktionary}
[...]
\end{verbatim}

\section{Degrees}

\begin{mdframed}
\texttt{\textbackslash Degree\{YEAR\}\{NAME-OF-DEGREE\}\{NAME-OF-INSTITUTION\}\newline
\hspace*{2cm}[\emph{THESIS-TITLE}]<\emph{MATH-GENEOLOGY-URL}>}
\end{mdframed}

Note the varied delimiters for different components; the two arguments
listed in the second line are optional.  (But to assist with automated
processing, the command should be entered on one line.)

\subsection*{Examples}

In the most typical case, all arguments will be provided.  The URL
will be formatted as a link.

\begin{verbatim}
\Degree{1979}{Ph.D.}{Miskatonic University}[A Statistical
  Interrogation of `Psychological Distress and Repeated Television
  Viewing']<http://genealogy.math.ndsu.nodak.edu/id.php?id=123456>
\end{verbatim}

\Degree{1979}{Ph.D.}{Miskatonic University}[A Statistical
  Interrogation of `Psychological Distress and Repeated Television
  Viewing']<http://genealogy.math.ndsu.nodak.edu/id.php?id=123456>

However, for some reason the geneology record is missing, we
can just provide the title:

\begin{verbatim}
\Degree{1979}{Ph.D.}{Miskatonic University}[A Statistical
Interrogation of `Psychological Distress and Repeated Television
Viewing']
\end{verbatim}

\Degree{1979}{Ph.D.}{Miskatonic University}[A Statistical
Interrogation of `Psychological Distress and Repeated Television
Viewing']

Similarly, if the thesis title is missing, we can just provide the
geneology record:

\begin{verbatim}
\Degree{1979}{Ph.D.}{Miskatonic
University}<http://genealogy.math.ndsu.nodak.edu/id.php?id=123456>
\end{verbatim}

\Degree{1979}{Ph.D.}{Miskatonic University}<http://genealogy.math.ndsu.nodak.edu/id.php?id=123456>

Finally, some types of degree don't expect either of those optional
arguments, or they may not be known, and in such cases, they can both
be left off.

\begin{verbatim}
\Degree{1975}{B.A.}{Miskatonic University}
\end{verbatim}

\Degree{1975}{B.A.}{Miskatonic University}

\section{Career}

\begin{mdframed}
\texttt{\textbackslash Position\{YEAR(S)\}\{ROLE\}\{INSTITUTION\}}

\texttt{\textbackslash Service\{YEAR(S)\}\{ROLE\}\{INSTITUTION\}}

\texttt{\textbackslash Affiliation\{INSTITUTION\}}\{\emph{URL}\}
\end{mdframed}

Only a few legacy users specify an \verb|\Affiliation{...}| and they
generally do so with one argument, which is sometimes the name of an
institution, sometimes a URL, or sometimes both together.  Here we
list one required argument which is meant to denote the institution,
and introduce \emph{url} as an optional second argument.

\subsection*{Examples}
\begin{alltt}
\verb|\subsection*{Career}|
\verb|\Position{|1979--1989\verb|}{|Researcher\verb|}{|Amundsen-Scott South Pole\(\rightarrow\)
    Station\verb|}|
\verb|\Position{|1989--2001\verb|}{|Professor of Statistics\verb|}|\verb|{|Miskatonic\(\rightarrow\)
    University\verb|}|
\verb|\Affiliation{PlanetMath.org}{http://planetmath.org}|
\verb|\subsection*{Professional Service}|
\verb|\Service{1990-1992}{Correspondent}{London| Psychogeographical\(\rightarrow\)
    Association\verb|}|
\end{alltt}
\vspace{-.5cm}
\subsubsection*{Career}
\Position{1979--1989}{Researcher}{Amundsen-Scott South Pole Station}
\Position{1989--2001}{Professor of Mathematics}{Miskatonic University}

\Affiliation{PlanetMath.org}{http://planetmath.org}
\vspace{-.3cm}
\subsubsection*{Professional Service}
\Service{1990-1992}{Correspondent}{London Psychogeographical Association}

\section*{Honors}

\begin{mdframed}
\texttt{\textbackslash Honor\{YEAR\}\{INSTITUTION/NAME-OF-HONOR\}\{\emph{CITATION}\}}
\end{mdframed}

The \emph{citation} field is optional.

\subsection*{Examples}

\begin{alltt}
\verb|\Honor{2001}{Fellowship of the Ring}|
\verb|\Honor{2015}{League of Extraordinary Gentlemen}{Elected|\(\rightarrow\)
    posthumously for services to literature.\verb|}|
\end{alltt}

\Honor{2001}{Fellowship of the Ring}
\Honor{2015}{League of Extraordinary Gentlemen}{Elected posthumously for services to literature.}

\section*{Membership in other organizations}

\begin{mdframed}
\texttt{\textbackslash Member\{INSTITUTION\}}
\end{mdframed}

Note, this field is largely unused and likely to be deprecated.  It is
included here only for historical compatibility.

\subsection*{Example}

\begin{verbatim}
\Member{Association of Autonomous Astronauts}
\end{verbatim}

\Member{Association of Autonomous Astronauts}


\end{document}

